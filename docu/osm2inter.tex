\part{osm2inter}
\chapter{osm2pgsql}
\section{Intro}
Als Importierungskomponente kann osm2pgsql\cite{osm2pgsql-manual} sehr vielseitig eingesetzt werden. Innerhalb des Projektes löst osm2pgsql den OHDMConverter zum importieren von osm\textbar osm.pbf\textbar osm.bz2 in die intermediate Datenbank ab.\\

Den größten Vorteil von osm2pgsql bietet die Benutzung des \gequote{Flex Output}. Hierbei wird die Konvertierung mit einem lua Script angepasst.

\section{osm2pgsql Flags}
Für die fehlerfreie Importierung sind mehrere Flags notwendig. In der nachfolgenden \autoref{tb:flags} sind diese vermerkt.
\begin{table}[h]
	\caption{flags}
	\label{tb:flags}
	\renewcommand{\arraystretch}{1.5}
	\begin{tabularx}{\linewidth}{|l|X|}\hline
		Flag & Beschreibung\\\hline
		-c & Spezifiziert die osm Datei\\\hline
		-d & Name der Datenbank \\\hline
		-U & Name des Datenbankbenutzers mit Rechten zur Erstellung (Standard: postgres)\\\hline
		-O & Spezifiziert den Output z.B.: flex, pgsql (Standard), gazetteer und null\newline Für die Benutzung des lua Script basierten konvertierens, wird dieser Wert auf flex gesetzt\\\hline
		-x & Ermöglicht die Verwendung von user name, user id, changeset id, timestamp and version\\\hline
		-S & Dies gibt an, wie die Daten in die Datenbank importiert werden, ihr Format hängt von der Ausgabe ab\newline in diesem Flag muss das lua Script angegeben werden\\\hline		
	\end{tabularx}\vspace{0.5cm}
In Windows muss zusätzlich -W für eine Passworteingabe als Flag gesetzt werden
\end{table}

\todo[inline]{Befehl mit \lstinline|sudo -iu postgres| und absoluten pfaden}
Ein Importierungsbefehl könnte nun wie folgt aussehen:
\begin{lstlisting}
	osm2pgsql -d ohdm -U postgres -O flex -x -S osm2inter.lua -c berlin.osm
\end{lstlisting}
\todo[inline]{Anleitung psql}
