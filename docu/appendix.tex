%\setcounter{page}{1} \addtocontents{toc}{\protect\pagebreak}
\part{\appendixname}
\appendix
\ihead{\thepart. \parttitle} \chead{} \ohead{\headmark}
\ifoot{\printTitle} \cfoot{Seite \pagemark} \ofoot{}

\newpage
\setheadsepline{.5px}
\setfootsepline{.5px}
\chapter{Multiple Datenbanken in PostgreSQL}\label{ch:clustering}
Die Alternative zur Erstellung eines PostgreSQL Clusters\cite{postgresql-cluster} ist die Verwendung von \lstinline[language=bash]|initdb|\cite{postgresql-cluster}, allerdings gibt mit dieser Variante einige Herausforderungen die mit \lstinline[language=bash]|pg_createcluster| leichter beziehungsweise überhaupt zu bewältigen waren.
\begin{enumerate}
	\item Steuerung des Clusters\cite{postgresql-cluster} für die Serververwaltung
	\item Cluster als Service auch nach einem Neustart des Server starten
\end{enumerate}

\chapter{Curl Map Features}\label{ap:ch:curl-mapfeatures}
Um die Arbeit mit den Map Features\cite{osm-mapfeatures} zu erleichtern, müsste man ein curl Skript implementieren, dass die Tabelleneinträge auf der Map Features\cite{osm-mapfeatures} Webseite ausliest und in eine csv Datei oder ähnliches schreibt.\\

Damit wäre es im Anschluss möglich die csv Datei als Grundlage für ein Insert Statement der classification Tabelle zu verwenden.
