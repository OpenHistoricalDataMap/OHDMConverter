\setcounter{page}{1} \addtocontents{toc}{\protect\pagebreak}
\part{\appendixname}
\appendix
\ihead{} \chead{} \ohead{} \ifoot{} \cfoot{\pagemark} \ofoot{}

\newpage
\setheadsepline{.5px}
\setfootsepline{.5px}
\ohead{\headmark}
\chapter{Multiple Datenbanken in PostgreSQL}\label{ch:clustering}
Die Alternative zur Erstellung eines PostgreSQL Clusters ist die Verwendung von \lstinline[language=bash]|initdb|\cite{postgresql-cluster}, allerdings gibt mit dieser Variante einige Herausforderungen die mit \lstinline[language=bash]|pg_createcluster| leichter beziehungsweise überhaupt zu bewältigen waren.
\begin{enumerate}
	\item Steuerung des Clusters für die Serververwaltung
	\item Cluster als Service auch nach einem Neustart des Server starten
\end{enumerate}

\chapter{Windows Flag W}\label{ap:ch:win-pass}
Theoretisch sollte es möglich sein auch in Windows eine Peer Authentifizierung zuzulassen. Dafür müsste die Einstellung in der \lstinline|pg_hba.conf| geändert werden und dann validiert ob und wie mit der Datenbank interagiert werden kann.

\chapter{Curl Map Features}\label{ap:ch:curl-mapfeatures}
Um die Arbeit mit den Map Features\cite{osm-mapfeatures} zu erleichtern, müsste man ein curl Script implemntieren, dass die Tabelleneinträge auf der Map Features Webseite ausliest und in eine csv Datei oder ähliches schreibt.\\

\noindent Damit wäre es im Anschluss möglich die csv Datei als Insert Grundlage der classification Tabelle zu verweden.
